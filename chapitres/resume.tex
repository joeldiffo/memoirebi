\chapter*{Résumé}%
\addcontentsline{toc}{chapter}{\numberline{}Résumé}%
Les nombreux avantages qu’apportent les outils de Business intelligence (BI) en entreprise nous dirige vers une ère où la décision en entreprise sera nettement amélioré à travers l’utilisation de la BI. Les entreprises tel que AMD Sarl disposent de grandes quantités de données inexploités qui peuvent apporter une plus-value inestimable a l’entreprise. Ces données peuvent être consolidées ainsi que des données commerciales provenant d’autres sources dans un entrepôt de données qui sera ensuite exploité à travers des outils de visualisation de données qui auront pour but de représenter ces données agrégées sous formes graphique facilitant ainsi énormément la compréhension pour les décideurs et mais aussi les commerciaux de l’entreprise. Nous avons décidé de mettre notre système de BI avec l’ensemble d’outils Microsoft Business Intelligence qui incorpore SSIS (SQL Server Integration Service), SSAS (SQL Server Analysis Service), SSRS (SQL Server Reporting Service). Le SGBD SQL Server a été utilisé pour mettre sur pied notre entrepôt de données et pour la visualisation nous avons opté pour Microsoft Power BI.

\paragraph{Mots-clefs :}Business intelligence