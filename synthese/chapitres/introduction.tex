\chapter*{Introduction}

\paragraph{}
Durant une longue période, la prise des décisions et la mise en place des stratégies au sein d’une entreprise étaient réalisées selon l’intuition de la Direction Générale et sans l’aide de l’informatique. Cela était dû au fait que les logiciels informatiques de l’époque ne permettaient pas la récupération des données à partir des applications transactionnelles ni de faire des calculs complexes essentiels pour la génération des rapports synthétiques sur l’activité.
\paragraph{}
Avec le développement informatique et l’apparition des éditeurs spécialisés dans les systèmes d’informations, les projets Business Intelligence (de la Data Warehouse jusqu’à la Data Visualisation) envahissaient petit à petit le monde de l’entreprise.
\paragraph{}
En effet grâce aux outils BI, il est devenu possible d’extraire plus facilement des gros volumes de données à partir de différentes sources, de les consolider dans un même entrepôt de données (Data Warehouse) et de les traiter profondément pour fournir aux décideurs comme aux métiers, les statistiques et les informations pertinentes dont ils ont besoin. Ces indicateurs clefs permettront de mieux comprendre la situation de l’entreprise, de mettre en œuvre la meilleure stratégie et de piloter d’une manière plus efficace l’activité de la société.
\paragraph{}
Les projets décisionnels, représentent ainsi une véritable opportunité pour les entreprises qui se précipitaient pour intégrer les outils BI dans leurs systèmes d’information et bénéficier ainsi d’une plus-value certaine non seulement en terme du temps mais aussi en termes d’argent. Nous faisons ici la ynthèse  d'un travail qui s’inscrit dans ce contexte. 
\paragraph{}
Le travail est divisé en 5 chapitres. Le chapitre 1 qui donne un état de l’art, le chapitre 2 contient le cahier de charges, le chapitre 3 contient l'analyse du projet, le chapitre 4 contient la conception du datawarehouse et le chapitre 5 contient l'implémentation et l'exploitation. A la fin une conclusion est faite et des perspectives futures mentionnées.
