\chapter*{Développement du Mémoire}

\section*{Chapitre 1 : Etat de l'Art}
Cette partie a permit d'essayer de comprendre les différents concepts qui seront abordés tout au long de la réalisation du projet. Commençant par les systèmes décisionnels ou encore l’intelligence des affaires (Business Intelligence en Anglais), ensuite les entrepôts de données, les ETL, les magasins de données, les cubes OLAP, les systèmes de reporting (la visualisation des données) et enfin la gestion commerciale. La liste suivante contient les différents notions abordés pour accoomplir l'objectif de ce chapitre :

\begin{itemize}
    \item Systèmes décisionnels ou « Business Intelligence »
    \item Les entrepôts de données ou « Datawarehouse »
    \item Les processus ETL
    \item Les magasins de données ou « datamarts »
    \item Les cubes OLAP
    \item Les systèmes de « reporting »
    \item Gestion commerciale et décisionnel
\end{itemize}



\section*{Chapitre 2 : Cahier de Charges}
Comme son nom ce chapitre avait pour but de ressortir les différents éléments présents dans le cahier de charges d’un projet. Nous avons commencé par étudier la problématique dont fait face l’entreprise et de là nous avons ressorti les objectifs du projet. Ensuite nous avons abordé les besoins, les fonctionnalités et les contraintes. Puis nous avons terminé par la planification et l'estimation des couts.

Le contenu de ce chapitre est comme suit : 
\paragraph{Contexte :} Parle du contexte dans lequel on se trouvait pour le déroulement du projet.

\paragraph{Problématique :} Il était question ici de ressortir exactement ce que l’entreprise fait face comme problème. Elle contient deux parties :
\begin{itemize}
    \item La démarche d’analyse du problème
    \item L'identification de la problématique
\end{itemize} 

\paragraph{Objectifs} : Après avoir posé les problèmes, il était question de parler des objectifs du projet. Divisé en deux parties :
\begin{itemize}
    \item Objectif général
    \item Objectifs spécifiques
\end{itemize}

\paragraph{Besoins :} Ensuite il fallait parler des besoins du client. Ainsi cette partie a été divisé en deux :
\begin{itemize}
    \item Besoins fonctionnels
    \item Besoins non-fonctionnels
\end{itemize}

\paragraph{Fonctionnalités attendus} : Connaisant les besoins on pu ressortir globalement les fonctionnalités attendus de la solution.

\paragraph{Contraintes} : Ensuite les contraintes ont étés abordés. Il y'en avait dans deux catégories : Technologiques et temporelles.

\paragraph{Intervenants du projet} : C'était la prochaine rubrique ou les les intervenants du projet ont étés cités avec leurs roles dans le projet.

\paragraph{Planification du projet} : Cela fait il fallait maintenant planifier le projet. Ceci a été fait découpant le projet en phases et tachesqui ont étés représentés dans un tableau avec leurs durées et précédences correspondants. Ceci a été représenté sur un diagramme de Gantt pour une meilleure visibilité et un meilleur suivi.

\paragraph{Estimation des coûts} : Le chapitre s'est terminé avec une estimation des couts divisé en deux comme suit :
\begin{itemize}
    \item \textbf{Méthode d’estimation des coûts}, ou les méthodes d'estimation ont étudiés et la meilleure choisi
    \item \textbf{Coût du Projet}, ou le projet a été évalué.
\end{itemize}


\section*{Chapitre 3 : Analyse du Projet}
Dans ce chapitre une analyse du projet a été effectué. Nous avons commencé par parler de la méthode utilisé pour le déroulement du projet, les outils et techniques pour le bon déroulement du projet. Puis est venu une analyse de l'existant. Ensuite l’aspect analyse fonctionnelle du système à mettre sur pied a été abordé d’où est ressorti une solution qu’on a présenté. Elle comporte les parties suivantes:
\paragraph{Gestion du projet :} Ici nous avons abordés les sujets suivants :
 \begin{itemize}
     \item Processus de développement
     \begin{itemize}
         \item Activités de développement des SI
         \item Importance du processus de développement des SI
         \item Types de processus de développement
     \end{itemize}
     \item Outils et techniques de gestion
     \begin{itemize}
         \item Présentation du processus de développement agile : SCRUM
         \item Outils collaboratifs
     \end{itemize}
 \end{itemize}

\paragraph{Analyse de l’existant :} Il était question de faire une analyse approfondi du système en place afn de comprendre au mieux comment mettre en place une solution. Fait en trois parties :
\begin{itemize}
    \item La gestion commerciale à AMD
    \item Critique de l’existant
    \item Les solutions concurrentes
\end{itemize}

\paragraph{Analyse Fonctionnelle }: A permis de faire l'analyse complète du système a mettre en place. Contient les tires suivants: 
\begin{itemize}
    \item La démarche d’analyse fonctionnelle
    \begin{itemize}
        \item La méthode APTE
        \item Les étapes de l’A.F.
        \item L'Analyse du Besoin (A.B.)
        \item L'Analyse Fonctionnelle du Besoin
        \item L'Analyse Fonctionnelle Technique
    \end{itemize}
\end{itemize}

\paragraph{Solution Retenue} : Ici nous avons présentés la solution avec une brève description

\section*{Chapitre 4 : Conception du Datawarehouse}
Ce chapitre a été dédié à la conception de l’entrepôt de données du système de business intelligence. Nous avons commencé par présenter entièrement l’architecture du système à concevoir, puis nous avons concu chacune des parties qui constituent la mise en place d’un entrepôt de données. On a débuté par le datawarehouse, duquel a découlé les datamarts. Ensuite on est passé à la conception des ETL qui seront chargés d’alimenter notre datawarehouse.

\paragraph{Architecture du système de Business Intelligence} : Ici nous avons présentéde facon globale l'architecture du système en entier, Commençant par les sources multiples des données, passant par les ETL, le datawarehouse, les datamarts et les cubes puis les visuels.

\paragraph{Conception du Datawarehouse et des Datamarts} : Dans cette section nous avons concu les différentes base de données que comportera notre système. Commençant par la base de données STAGING qui est une base de données de preparation de données pour chargement dans l'entrepôt de données final. Puis nous avons concu la DATAWAREHOUSE véritablement dite. Ensuite la dernière phase était la conception des différents DATAMARTS a partir du DATAWAREHOUSE. 

\paragraph{Conception des ETL} : Ici nous avons commencé par présenter l'architecture d'un processus ETL. Nous avons ensuite parlé de l'extraction des données et de commment les données devaient etre extraites des différentes bases opérationnelles. Ensuite on a fait une analyse sur les transformations que devaient subir les données pour les rendre uniforme dans l'entrepôt. Puis les différentes méthodes de chargement dans la DATAWAREHOUSE ont étés abordées pour finaliser la conception des ETL pour alimenter notre DATAWAREHOUSE. La section a été cloturé par une revue sur comment l'ETL sera utilisé et une présentation des schemas des ETL concus. 


\section*{Chapitre 5 : Implémentation et Exploitation}
Chapitre final du document, il contient des éléments sur l'implémentation du système et son exploidata avec les outils de data visualisation. Nous avons présenté les ETL conçus, l’entrepôt de données et les magasins de données, les cubes d’analyse conçus et quelques tableaux de bords réalisés. Nous avons commmencé par faire un choix d'outil d'implémentation du système. D'ou on a choisi d'implémenter notre système avec la SGBD MSSQL Server. La prochaine étape était de décrire le processus d'installation de SQL Server 2019, la version la plus récente du logiciel. Ensuite nous sommes passés a la création des bases de données en commençant par la STAGING, puis la DATAWAREHOUSE ensuite les DATAMARTS. 

Après c'était le moment de concevoir les ETL vu qu'ils dépendent fortement de la base dans laquelle elle aura la tache de charger les données. Nous avons commencé par faire un choix de l'outil a utiliser pour l'implémentation. De cette étape est ressorti l'outil SSIS du pack d'outils Microsoft Business Intelligence. Nous avons ensuite vu comment télécharger et installer SSIS. La prochaine étape logique était de passer a la création des ETL avec l'outil que nous venions d'installer. 

Nous avons commencé par créer l'ETL de chargement de données dans la STAGING. C'est l'ETL qui avait pour travail d'extraire les données des différentes sources et les agréger dans la STAGING d'ou ils seront chargés dans la DATAWAREHOUSE. Ensuite nous avons crée l'ETL d'alimentation de la DATAWAREHOUSE a partir de la STAGING, puis l'ETL de chargement des différents DATAMARTS. L'alimentation de la DATAWAREHOUSE et des DATAMARTS se fait en éxécutant simplement des procédures stockées que nous avions crées au préalable dans notre DATAWAREHOUSE. 

La prochaine étape était de charger les données qui a été fait a partir du connecteru ODBC et une connexion aux fichiers plats puisque ce sont les types correspondants de nos sources de données. 

Après ceci c'était l'étape d'exploitation de la DATAWAREHOUSE. Cette partie a commmencé par le choix de l'outil a utiliser. Nous sommes allés avec Microsoft Power BI. Ensuite nous avons utilisé le module d'analyse de Power BI pour créer nos cubes de données ce qui facilitera les représentations sur les visuels. Avec nos cubes nous avons donc crée nos tableaux de bords qui présentent des graphiques facilement interpretables. La section et le chapitre se termine par le déploiement des tableaux de bords sur Power BI Service d'ou le client peut accéder avec son navigateur ou son téléphone mobile.


\paragraph{}
Le document se cloture avec une conclusion générale du projet et quelques perspectives futures sur l'expansion du système nouvellement construit.

