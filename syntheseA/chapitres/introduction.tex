\chapter * {Introduction}

\paragraph {}
For a long time, decision-making and strategy implementation within a company were carried out on the intuition of top management and without the help of IT. This was because computer software at the time did not allow data retrieval from transactional applications or the complex calculations essential for generating synthetic activity reports to be performed.
\paragraph {}
With IT development and the emergence of publishers specializing in information systems, Business Intelligence projects (from Data Warehouse to Data Visualization) gradually invaded the business world.
\paragraph {}
Indeed, thanks to BI tools, it has become possible to more easily extract large volumes of data from different sources, to consolidate them in the same datawarehouse and to process them deeply to provide decision-makers as well as commercials, statistics and the relevant information they need. These key indicators will make it possible to better understand the situation of the company, to implement the best strategy and to manage the activity of the company more effectively.
\paragraph {}
Business Intelligence projects thus represent a real opportunity for companies who will rush to integrate BI tools into their information systems and thus benefit from an assured added value not only in terms of time but also in terms of money. The work carried out for this thesis fits into this context.
\paragraph {}
The work is divided into 5 chapters. Chapter 1 which gives a state of the art, chapter 2 contains the specifications, chapter 3 contains the analysis of the project, chapter 4 contains the design of the datawarehouse and chapter 5 contains the implementation and exploitation. At the end a conclusion is made and future prospects mentioned.