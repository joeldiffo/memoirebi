\chapter*{Introduction}%
\addcontentsline{toc}{chapter}{\numberline{}Introduction}%
Avec le développement informatique et l’apparition des éditeurs spécialisés dans les systèmes d’informations, les projets Business Intelligence (de la Data Warehouse jusqu’à la Data Visualisation) envahissaient petit à petit le monde de l’entreprise.
\paragraph{}
Grâce aux outils BI, il est devenu possible d’extraire plus facilement des gros volumes de données à partir de différentes sources, de les consolider dans un même entrepôt de données (Data Warehouse) et de les traiter profondément pour fournir aux décideurs comme aux métiers, les statistiques et les informations pertinentes dont ils ont besoin. Ces indicateurs clefs permettront de mieux comprendre la situation de l’entreprise, de mettre en œuvre la meilleure stratégie et de piloter d’une manière plus efficace l’activité de la société.\paragraph{}
Les projets décisionnels, représentent ainsi une véritable opportunité pour les entreprises qui se précipitaient pour intégrer les outils BI dans leurs systèmes d’information et bénéficier ainsi d’une plus-value certaine non seulement en terme du temps mais aussi en termes d’argent. C'est dans ce contexte que ce stage en \textbf{Développement Business Intelligence} s'inscrit ou nous avons travaillé sur un projet de mémoire intitulé \textbf{Optimisation de la Gestion Commerciale d’une Entreprise avec un Datawarehouse : Cas de AMD Sarl.}