\chapter*{Avant Propos}%
\addcontentsline{toc}{chapter}{\numberline{}Avant Propos}%

\par Dans le but d’assurer son développement et de fournir à ses industries des agents compétents dans divers domaines, l’Etat camerounais a créé de nombreuses structures et formation parmi lesquelles le BTS (Brevet de Technicien Supérieur) crée par arrêté ministériel n°90/E/ (58MINESUP/DUET du 18décembre1971. L’institut Supérieur des Techniciens et du Design Industriel a été créé par arrêté N°02 /0094/MINESUP/ESUP. DDEES/ESUP du 18 septembre 2002. L’ISTDI a été érigé plus tard en Institut Universitaire de la cote(IUC) par arrêtée N°01/05156/N/DDES/ESUP/SAS/EBM/du 21 décembre 2011. L’IUC porte trois établissements.\\
	\begin{itemize}
        \item \textbf{L’institut supérieur des technologies et du design industriel (L’ISTDI)} qui forme dans les cycles et filières suivantes :
        \begin{itemize}
            \item \textbf{CYCLE BTS INDUSTRIEL}
            \item \textbf{CYCLE DES LICENCES PROFESSIONNELLES}
            \item \textbf{CYCLE MASTER PROFESSIONNEL}
        \end{itemize}
        \item \textbf{L’INSTITUT DE COMMERCE ET D’INGENIERIE D’AFFAIRES (ICIA)}
        \begin{itemize}
            \item \textbf{CYCLE BTS COMMERCIAUX}
            \item \textbf{CYCLE LICENCE PROFESSIONNELLES COMMERCIALES}
            \item \textbf{CYCLES MASTER PROFESSIONNELLES ISUGA – France}
        \end{itemize}
        \item \textbf{L’INSTITUT D’INGENIERIES INFORMATIQUES D’AFRIQUE CENTRALE(3IAC)} qui forme dans les cycles et filières suivantes :
        \begin{itemize}
            \item \textbf{Cycle des NTIC}
            \item \textbf{Cycle master Européen}
            \item \textbf{Cycle ingénieur}
        \end{itemize}
        \item \textbf{4- PISTI (Programmes Internationaux de Sciences et Technologies de l’Innovation)} comprend les filières ci-dessous:
        \begin{itemize}
            \item Classes préparatoires aux grandes écoles d’ingénieur
            \item ESSTIN (école supérieure des sciences et technologies de l’ingénieur de Nancy-France) avec Nancy L’étudiant fait un cycle de 03 ans au Cameroun et obtient un BACHELOR puis 02 ans à l’université de Nancy-France.
            \item LST (Licences sciences et techniques) avec l’université du Maine. L’étudiant fait un cycle de 03 ans au Cameroun, puis 02 ans à l’université du Maine en France.
            \item ADI (architecture et design industriel) avec l’université de Rome 1 de Camerino-Italie. L’étudiant désireux fait un cycle de 02 ans au Cameroun et par la suite termine avec 03 ans en Italie.
            \item IBM (ingénierie biomédicale) avec l’université de Rome2 de Tor Vergata-Italie.L’étudiant désireux fait un cycle de 03 ans au Cameroun et par la suite termine avec 02 ans en Italie.
        \end{itemize}
    \end{itemize}
	
	\par L’étudiant du cycle ingénieur est tenu en 5ème année d’effectuer un stage académique d'une durée de 06 mois. Ce stage permettra aux étudiants de mieux appréhender les connaissances acquises et surtout de les appliqués à un domaine de la spécialité. C’est dans cette optique que nous avons effectué un travail sur le thème : \textbf{Optimisation de la Gestion Commerciale d'une Entreprise avec un Datawarehouse: Cas de AMD Sarl}.
