\chapter{Déroulement du Stage}

\section{Accueil}
Notre arrivée a INTERFACE a été marque par un accueil chaleureux de la part du chef de projet, du chef d’équipe de Développement Business Intelligence (B.I.) et du reste des membres de l’équipe B.I. déjà là, a notre arrivée. Nous avons étés présentés aux différentes équipes de développement ainsi qu’aux autres employés dans le reste des départements de l’entreprise. Nous n’avions pas beaucoup de temps à perdre puisque le projet courant de l’équipe B.I. était à un point critique. Donc nous nous sommes directement lancés dans le vif du sujet, la prise en main des technologies. 

\section{Prise en main des technologies de l'entreprise}
\subsection{Objectif}
Comme indiqué dans le premier chapitre INTERFACE S.A. a un pôle Software Engineering (Génie Logiciel). Ce pole contient une équipe de développement de solutions de Business Intelligence et pour notre travail, INTERFACE depuis longtemps utilise la suite des outils de Business Intelligence de Microsoft, Microsoft Business Intelligence Tools qui comporte des outils allant de l’extraction de données jusqu’au reporting.

L’objectif de la prise en main des technologies était pour que nous les nouveaux stagiaires puissent maitriser l’utilisation routinière des outils en question pour effectuer les taches qui nous seront attribuées par la suite.

\subsection{Méthodologie de travail}
Nous avons étés lancés dans le bain rapidement par le chef de projet en nous faisant passer une formation de 2 semaines sur les technologies SSIS (SQL Server Integration Services) et SSRS (SQL Server Reporting Services). Ces deux technologies nous aideront tout au long du déroulement du stage. Les ainées de l’équipe étaient permanemment présents pour nous guider et nous débloquer rapidement, histoire d’éviter la perte de temps.
\paragraph{}
La formation comportait des taches similaires a celles qui nous seront attribues dans le projet, avec des taches réelles mais un peu basique attribuées de temps en temps.



\subsection{Enseignements retirés}
Nous allons les citer sous forme de tirets
\begin{itemize}
    \item La maitrise d’un outil de Business intelligence largement utilisé dans l’industrie.
    \item Une méthode de travail en entreprise pour les projets de Business Intelligence
    \item Quelques notions dans l’industrie des assurances, puisque c’est le domaine du client et on devait comprendre un minimum son fonctionnement pour lui fournir une solution
\end{itemize}

\section{Le projet CIMA}

\subsection{Résumé du projet}
La CIMA (Conférence Interafricaine des Marches d’Assurances) est l’agence de règlementation et de supervision des entreprises d’assurances en Afrique subsaharienne. 17 pays Africains sont affiliées a la CIMA.
\paragraph{}
Le projet a pour intitule \textbf{\textit{"Mise en œuvre d’une plateforme numérique de centralisation des états réglementaires et d’aide à la supervision à la CIMA"}} et a pour objectif de reproduire les différents rapports (états) crées par les entreprises a l’issu de leurs activités. Ceci pour pouvoir faire des tests de conformités et de cohérences assurant ainsi le respect des règlements mises sur place par la CIMA ar ces entreprises. Aussi cette plateforme permettra à la CIMA de consulter les états des marchés pour toutes les périodes souhaitées à travers les rapports annuels que nous produirons.

\subsection{Méthodologie de travail}
Notre travail était constitué généralement de deux taches principales. 
\begin{itemize}
    \item La mise en place de l’ETL (L’outil permettant l’extraction et l’agrégation des données et leurs chargements dans l’entrepôt de données) à partir d’un Template d’ETL pour l’alimentation de l’entrepôt de données.
    \item La production des rapports (états) pour déploiement sur leur plateforme
\end{itemize}

Le projet CIMA est un projet qui dure depuis Juin 2019 et comporte des phases majeures dont nous les stagiaires n’avons pas fait partie de l’implémentation notamment :
\begin{itemize}
    \item La mise sur pied de l’entrepôt de données et tous ses composants
    \item La mise sur pied de la plateforme numérique pour la consultation des états que nous produirons
    \item Le design des Template d’ETL et de rapport utilises pour produire le reste
\end{itemize}
Des réunions en ligne avec le client (Commissaire de la CIMA) situé au Gabon, au siège de la CIMA et des représentants des différentes entreprises se faisaient de façon journalier ou hebdomadaire dépendant du planning du projet. Un stagiaire pouvait être appelé à participer à cette réunion s’il avait travaillé sur les états consultés pendant la réunion. 


\subsection{Solution apportée}

Le résultat de ce projet est une plateforme numérique qui centralise les états réglementaires des entreprises d’assurances dans les pays affilies a la CIMA et qui aide à la supervision de ces différents états pour assurer la cohérence et la conformité des règles de la CIMA. Le projet est en phase de terminaison et doit être livré d’ici le mois d’Octobre.

\subsection{Enseignements retirés}
On peut les distinguer dans les tirets suivants : 
\begin{itemize}
    \item Le projet CIMA a permis une montée en compétence chez nous les stagiaires dans le domaine de la Business Intelligence
    \item L’expérience acquise sur les méthodologies de travail dans les projets d’envergure
    \item Aussi, nous en ressortirons de ce projet avec beaucoup de notions dans les marches d’assurances, ce qui est toujours un plus pour la culture générale.
\end{itemize}


\section{Travail du Mémoire}

\subsection{Résumé du projet}
L'entreprise AMD Sarl encontrait des problèmes de fixation des prix de leurs produits pour optimiser les ventes à certaines périodes. Aussi ils faisaient face au problème de ne pas pouvoir savoir de façon exacte le besoin de ses clients à certaines périodes, résultant ainsi a certaines commandes des clients ne pouvant pas être satisfaits. 
\paragraph{}
Nous avons donc proposé une solution de Business Intelligence pour l’entreprise qui leur propose des prix optimaux pour des produits a des périodes précises se basant sur l’étude des ventes du produit en question au fil des années. Aussi le système pourra renseigner les décideurs de l’entreprise sur les produits souvent demandés par chaque client sur une période en particulier. Ceci permettra à l’entreprise d’anticiper sur la demande et d’être prêt quand ça arrivera.

\subsection{Méthodologie de travail}
J’étais le seul ingénieur présent sur le développement de ce projet. Le chef de projet en entreprise était là pour le cadrage et les conseils à suivre pendant le développement du projet et le responsable de la recherche et de l’innovation à AMD Sarl a pleinement collaboré avec la fourniture de toutes les informations nécessaires à l’accomplissement du projet.
\paragraph{}
Nous avons procédé par la mise en place du cahier de charges puis l’étude de l’existant, ensuite l’analyse, la conception et la réalisation du projet. 
\paragraph{}
Les réunions sur l’évolution du projet se tenaient de façon hebdomadaire entre moi, le développeur et le responsable à AMD qui représentait le client. 

\subsection{Solution apportée}
Le résultat de ce travail est une plateforme de Business Intelligence contenant des tableaux de bords très riches et dynamiques pouvant renseigner les décideurs à AMD avec des informations critiques à la décision.
\paragraph{}
Cette plateforme est basée sur l’étude des données antérieurs de l’entreprise pour savoir l’état global des ventes et des commandes. 

\subsection{Enseignements retirés}

La création de cette plateforme m’a permis d’accentuer ma maitrise des outils de développement des système décisionnelles de bout en bout. Aussi j’ai eu l’expérience de pouvoir gérer un projet de bout en bout de la phase d’initiation jusqu’à la livraison du produit final. Ceci va accroitre mes compétences en tant qu’ingénieur. 


